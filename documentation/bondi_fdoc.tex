\documentclass[12pt]{article}
\usepackage{amsmath}
\usepackage{book_pc_no_env}
\usepackage{fancyvrb}
 \usepackage{simplemargins}
\usepackage{proof}
\usepackage{syntax}
\usepackage{sverb}
\usepackage{code}
\usepackage{enumitem}


\title{Growing \bondi\ One Class at a Time \\
Projects for Students of Programming with Patterns }

\author{Jose Vergara}
\date{}
\setleftmargin{1.5cm}
	\setrightmargin{1cm}
	\settopmargin{2cm}
	\setbottommargin{2cm}
\begin{document}
\maketitle
\makeatactive

\section{Introduction}

\bondi\ is a state-of-the-art programming language in which all
computation is based upon patterns. The current implementation is in
OCAML, a functional programming language. We are producing a \bondi\
interpreter in \bondi, with each language feature isolated in a single
object-oriented class. 

At this point we have implemented about half of the features. The rest
are available for you to work on. The typical project will require
development of one feature, i.e.\ one major class, along with its
documentation. 

This document describes the current state of the system; the intended
final state of the system; and the specifications of the various
features. 

\subsection{The Current State of the System}

The current state of the system is given in Figure~\ref{fig:comblang}.

\begin{figure}
  \begin{grammar}
<x> ::= i | n

<type> ::= type_var | t $\rightarrow$ t  | $\forall$x.t 

<exp> ::= <exp> <exp> | <operator> | exp_var | <sugar> 

<operator> ::= S | K | Y | I | F | E | B 

<sugar> ::= $\lambda x.$<e> | let  | let_rec | $p \; \rightarrow \; s$

<action> ::= shell_action | exp_action | let_action
\end{grammar}
\caption{The current combinatory language}
\label{fig:comblang}
\end{figure}


\subsection{The Intended State of the System}

The intended state of the system is given in Figure~\ref{fig:gcomblang}.

\begin{figure}
  \begin{grammar}
<x> ::= i | n

<type> ::= <type_var> | t $\rightarrow$ t  | $\forall$x.t | ref x | array x | <type_constant>

<type_constant> ::= Int | Float | Bool | String | Unit

<exp> ::= <exp> <exp> | <operator> | exp_var | <sugar> | Ref <expression> | constructor

<operator> ::= S | K | Y | I | F | E | B | new | newarray | lengthv | entry 
\alt  <bang> | <arithmetic_oper>  | infix_op

<sugar> ::= $\lambda x.$<e> | let  | let_rec | $p \; \rightarrow \; s$ | $[\theta]p \; \rightarrow \; s$ 

<action> ::= shell_action | exp_action | let_action | add_case | datatype_decl | class_decl 
             \alt   newarray | assignment | type_syn_decl | method_decl | static_method_decl

\end{grammar}  
\caption{The grown combinatory language}
\label{fig:gcomblang}
\end{figure}


\subsection{The List of Features to be Implemented}

Each item below represents a project for one student, unless stated
otherwise. Completion requires the ability to handle all of the bondi
test code, as given in the standard prelude, tests and pc_book. Other
tests may be added as necessary. 


The effort involved in each project will differ, and this
will be reflected in the assessment.  

Here are the new types and expressions to be added. Wildcards must be
handled in situ. For example, the wildcard for integers 
must be handled as a subclass of the integer class. 

\begin{enumerate}

\item{Types and Expressions}

\begin{enumerate}[label*=\arabic*]
\item numerical types, @Int@ and @Float@
\item alphabetical types @Char@ and @String@ and @Bool@
\item booleans and conditionals 
\item let-terms (with their status) 
\item dynamic patterns 
\item references 
\item arrays (including while- and for-loops) 
\item sockets (including file IO)
\item concurrent processes (cpc) (this is hard, and requires sockets)  
\end{enumerate}


\item{Shell Actions}
The new shell actions to be added are:
\begin{enumerate}[label*=\arabic*]
\item let-declarations (with let-status) 
\item linear declaration (to be made a let-status) 
\item adding a case
\item type synonyms 
\item datatype declarations
\item class declarations 
\item \ldots (including static methods and generic types)
\item augmenting declarations with trailing actions (with ... and ...)  
\end{enumerate}

\item{The Context}
The actions must be managed by a context.
\paragraph*{•}
The existing context must be
extended to support the following fields
\begin{enumerate}[label*=\arabic*]
\item
\begin{itemize}
\item globalTyEnv: ref (type_env);;
\item globalVEnv: ref (global_value_env) 
\item globalCEnv: ref (constructor_env) 
\item globalRefVarSub: ref sub where sub is a TyMap that produces types. 
\end{itemize}
\end{enumerate}

\paragraph*{}
In turn, these are implemented using various data structures. The OCAML
interpreter uses the following:

\begin{itemize}
\item{TyMap}: maps type variables to types, or other stuff: used widely.
\item{TMap}: maps term variables to types, or other stuff: used widely.
\item  type\_data: describes type synonyms and class data 
\item type\_env: a TyMap of type\_data. 
\item constructor\_env: a TMap producing indexed type and status data for constructors
\item scheme_env: a TMap producing  type and status data for variables 
\item global\_value\_env: and indexed mapping from variables to their types and values 
\end{itemize}

\item Data Sructures and Algorithms 
\paragraph*{}
In turn, the structures above require
\begin{enumerate}[label*=\arabic*]
\item
hash tables
\item
balanced search trees
\item garbage collection (hard) 
\end{enumerate}



\item Auxiliary Programs
\paragraph*{}
Here are features that are {\em not} within the natural class structure:
\begin{enumerate}[label*=\arabic*]
\item parsing of comments, infix operators, tuples and lists, etc.
\item the unit type, exceptions, etc 
\end{enumerate}


\item Integration and Testing
Integration and testing must work with the whole project team.  Marks
will be the average of the class as a whole, adjusted according to
feedback, or by negotiation. 
\begin{enumerate}[label*=\arabic*]
\item integration
\item testing
\end{enumerate}

\end{enumerate}


\section{Implementing Features}

One example of a language feature, is the K operator that has the following informal description:

Class: K

\begin{tabular}{|l|p{15cm}|}
\hline
Parsing & \begin{grammar}
<K> ::= `K'  (K is defined as the character `K')
\end{grammar} \\
\hline
Typing & $K \text{ has type }\forall X. \forall Y. X \rightarrow Y \rightarrow X$
\\
\hline
Evaluation & $K \;s\;t \;\;  \text{reduces to} \; s$ \\
\hline
Pretty printing & $K \; \text{has string} \; ``K"$ \\
\hline
\end{tabular} 

\paragraph*{•}
The description of Parsing is given ``Backus Naur Form" (BNF) \cite{grune2008parsing}.
The meta-symbols of BNF are:
\begin{tabular}{l l}
\verb@::=@ &  meaning ``is defined as" \\
\verb@|@    & meaning ``or" \\
\verb@< >@ & angle brackets used to surround category names \\
\verb@+,*,?@ & meaning ``one or more'',``zero or many'' and ``zero or one'' \\
\end{tabular} 
\paragraph*{}
Typing is given by mathematical symbols, the type scheme for the K operator,
Where $X$ and $Y$ are type variables, $X \rightarrow Y$ is a function type,
and $\forall.X.T$ is a quantified type.

\paragraph*{}
Evaluation is given by the reduction rule: ``$K$ applied to two arguments $s$ and $t$
reduces to $s$". 

\paragraph*{}
Pretty printing is displaying the string ``K"
\paragraph*{}
This informal description corresponds to the following \bondi\ class:
\scriptsize
\begin{Verbatim}[frame=single,framerule=0.2pt,framesep=6pt] 
class K_operator extends Operator {
static make =  { fun args -> Expression.make (new K_operator) args } 
(* parse "K" = K *)
static parser = { parseUpper "K" (K_operator.make []) }   
(*  K: forall X. forall Y. X -> Y -> X *)
type_scheme = { |() ->                                    
  let x = TypeVariable.next() in 
  let y = TypeVariable.next() in 
  QuantifiedType.make (x.get_variable(),
  (QuantifiedType.make (y.get_variable(),FunctionType.make2 (x,y,x) 
))) }
(* K x y -> x *)
reduce = { (( |() -> match this.get_arguments() with ((  
  | (Cons x (Cons y args)) -> Some (x.apply args)
  | _ -> None
): List Expression -> Maybe Expression)
): Unit -> Maybe Expression) }
(* toString K = "k" *)
headString = { |() -> "K" }                              
with 
Expression.simpleParser += | view(K_operator.parser,Parsed ps) -> Parsed ps }
\end{Verbatim}
\normalsize


\section{Types and Expressions}


\subsection{Numerical types, Int and Float}

\begin{tabular}{c}


\begin{tabular}{|l|p{15cm}|}
\hline
Class name & Integer extends Expression\\
\hline
Parsing & 
\begin{grammar}
<int> ::=  <digit>+ 
\end{grammar}  \\
\hline
Typing &  Int
\\
\hline
Evaluation &  has no evaluation rules \\
\hline
Pretty printing &  $t:T$  has string  ``it: $Int$   it = $t$" \\
\hline
\end{tabular} 

\\

\begin{tabular}{|l|p{15cm}|}
\hline
\hline
Class name & Integer_wildcard extends Expression\\
\hline
Parsing & 
\begin{grammar}
<int_wildcard> ::=  `_Int'
\end{grammar}  \\
\hline
Typing &  Int
\\
\hline
Evaluation &  has no evaluation rules \\
\hline
Pretty printing &  $\_Int$  has string  ``it: Int it = _Int" \\
\hline
\end{tabular} 

\\

\begin{tabular}{|l|p{15cm}|}
\hline
\hline
Class name & Float extends Expression \\
\hline
Parsing & 
\begin{grammar}
<float> ::=  <digit>+ `.' <digit>* 
\end{grammar}  \\
\hline
Typing &  Float
\\
\hline
Evaluation &  has no evaluation rules \\
\hline
Pretty printing &  $t:T$  has string  ``it: $Float$   it = $t$" \\
\hline
\end{tabular} 

\\

\begin{tabular}{|l|p{15cm}|}
\hline
\hline
Class name & Float_wildcard extends Expression\\
\hline
Parsing & 
\begin{grammar}
<float_wildcard> ::=  `_Float'
\end{grammar}  \\
\hline
Typing &  Float
\\
\hline
Evaluation &  has no evaluation rules \\
\hline
Pretty printing &  $\_Float$  has string  ``it: Float it = _Float" \\
\hline
\end{tabular} 

\end{tabular}

\paragraph{}
Besides the implementation of the Int and Float classes for expressions and types, 
it is necessary to implement  the following operators:

\subsubsection{Int operators}
\begin{enumerate}
\item  Integer Addition   		\textbf{plusint}
\item  Integer subtraction		\textbf{minusint}
\item  Integer multiplication   \textbf{timesint}
\item  Integer division         \textbf{divideint}
\item  Integer reminder         \textbf{modint}
\item  Integer unary negation   \textbf{negateint}
\item  Picks a random number between zero and an integer upper bound  \textbf{randomint};
\end{enumerate}

\subsubsection{Relational Int operators}
\begin{enumerate}
\item Less than   			  \textbf{lessthan int} 
\item Less than or equal   \textbf{lessthanorequal int} 
\item Greater than			  \textbf{greaterthan int} 
\item Greater than or equal  \textbf{greaterthanorequal int} 
\end{enumerate}

\subsubsection{Float operators}
\begin{enumerate}
\item  Float addition       \textbf{plusfloat} 
\item  Float substraction   \textbf{minusfloat} 
\item  Float multiplication \textbf{timesfloat} 
\item  Float division 		 \textbf{dividefloat} 
\item  Arc tangent x y returns the arc tangent of y / x \textbf{atan2}
\item  Float reminder   	\textbf{fmod} 
\item  Float pow, computes x raised to the power y  \textbf{pow}
\end{enumerate}

\subsubsection{Relational Float operators}
\begin{enumerate}
\item Less than   			  \textbf{lessthanfloat} 
\item Less than or equal   \textbf{lessthanorequalfloat} 
\item Greater than			  \textbf{greaterthanfloat} 
\item Greater than or equal  \textbf{greaterthanorequalfloat} 
\end{enumerate}

\subsubsection{Extra Float operators}
\begin{enumerate}
\item  Arc cosine 	\textbf{acos}
\item  Arc sine		\textbf{asin}
\item  Arc tangent  \textbf{atan}
\item  Round above to an integer value \textbf{ceil}
\item  Cosine  \textbf{cos}
\item  Hyperbolic cosine \textbf{hcos}
\item Exponential \textbf{exponential}
\item Asolute value \textbf{fabs}
\item Round below to an integer value {floor}
\item Natural logarithm \textbf{log}
\item Base 10 logarithm \textbf{log10}
\item Float unary negation  \textbf{negatefloat}
\item Picks a random number between zero and a float upper bound  \textbf{randomfloat}
\item Sin \textbf{sin}
\item Hyperbolic sine \textbf{sinh}
\item Square root \textbf{sqrt}
\item Tangent \textbf{tan}
\item Hyperbolic tangent \textbf{tanh}
\end{enumerate}



\paragraph{}
The PlusInt class is an example of the implementation of one of these operators:

\begin{tabular}{|l|p{15cm}|}
\hline
Class name & PlusInt extends Operator \\
\hline
Parsing & 
\begin{grammar}
<plusint> ::=  `plusint'
\end{grammar}  \\
\hline
Typing &  plusint has type  $Int \rightarrow Int \rightarrow Int$
\\
\hline
Evaluation &  $exp \; plusint \; exp$ reduces to $exp$\\
\hline
Pretty printing & \\
\hline
\end{tabular} 


\subsection{Alphabetical types Char and String}

\begin{tabular}{c}


\begin{tabular}{|l|p{15cm}|}
\hline
Class name & Char extends Expression \\
\hline
Parsing & 
\begin{grammar}
<char> ::=  `''<chatacter>`''
\end{grammar}  \\
\hline
Typing &  Char
\\
\hline
Evaluation &  has no evaluation rules \\
\hline
Pretty printing &  $t:T$  has string  ``it: $Char$   it = $'t'$" \\
\hline
\end{tabular} 

\\

\begin{tabular}{|l|p{15cm}|}
\hline
\hline
Class name & Char_wildcard extends Expression \\
\hline
Parsing & 
\begin{grammar}
<char_wildcard> ::=  `_Char'
\end{grammar}  \\
\hline
Typing &  Char
\\
\hline
Evaluation &  has no evaluation rules \\
\hline
Pretty printing &  $\_Char$  has string  ``it: Char it = _Char" \\
\hline
\end{tabular} 

\\

\begin{tabular}{|l|p{15cm}|}
\hline
Class name & String extends Expression \\
\hline
Parsing & 
\begin{grammar}
<string> ::=  `"'(<character>|<digit>)+`"'
\end{grammar}  \\
\hline
Typing &  String
\\
\hline
Evaluation &  has no evaluation rules \\
\hline
Pretty printing &  $t:T$  has string  ``it: $String$   it = $'t'$" \\
\hline
\end{tabular} 

\\

\begin{tabular}{|l|p{15cm}|}
\hline
\hline
Class name & Char_wildcard extends Expression \\
\hline
Parsing & 
\begin{grammar}
<char_wildcard> ::=  `_Char'
\end{grammar}  \\
\hline
Typing &  Char
\\
\hline
Evaluation &  has no evaluation rules \\
\hline
Pretty printing &  $\_Char$  has string  ``it: Char it = _Char" \\
\hline
\end{tabular} 

\end{tabular}

\subsubsection{Char operators}
\begin{itemize}
\item Char to Int conversion \textbf{char2int}
\end{itemize}

\subsubsection{Relational Char operators}
\begin{itemize}
\item Less than   			  \textbf{lessthanchar} 
\item Less than or equal   \textbf{lessthanorequalchar} 
\item Greater than			  \textbf{greaterthanchar} 
\item Greater than or equal  \textbf{greaterthanorequalchar} 
\end{itemize}

\subsubsection{String operators}
\begin{itemize}
\item String concatenation \textbf{concat}
\item String copy, return a copy of the given string \textbf{copystring}
\item Return a copy of the argument, with special characters represented by escape sequences \textbf{escapedstring}
\item String length \textbf{lengthstring}
\item Return a copy of the argument, with the first character set to uppercase \textbf{capitalizestring}
\item Return a copy of the argument, with all lowercase letters translated to uppercase \textbf{uppercasestring}
\item Converts a string into a boolean (i.e "True" is  coverted to True)  \textbf{string2bool}
\item Converts a string into an integer \textbf{string2int}
\item Converts a string into a float \textbf{string2float}
\item Returns character number n in string s \textbf{getcharstring}
\item Returns a fresh string of length n, filled with a character c \textbf{makestring}
\item Returns the character number of the first occurrence of character c in string s \textbf{indexstring}
\item Returns the character number of the last occurrence of character c in string s \textbf{rindexstring}
\item Tests if character c appears in the string s \textbf{containsstring}
\item Returns a fresh string of length len, containing the substring of s that starts at position start and has length len \textbf{substring}
\item Returns the character number of the first occurrence of character c in string s after position i \textbf{indexfromstring}
\item Tests if character c appears in s after position \textbf{containsfromstring}
\end{itemize}

\subsection{Booleans and conditionals}

\begin{tabular}{c}

\begin{tabular}{|l|p{15cm}|}
\hline
Class name & Bool extends Expression \\
\hline
Parsing & 
\begin{grammar}
<bool> ::=  `True' | `False'
\end{grammar}  \\
\hline
Typing &  Bool
\\
\hline
Evaluation &  has no evaluation rules \\
\hline
Pretty printing &  $t:T$  has string  ``it: $Bool$   it = $'t'$" \\
\hline
\end{tabular} 

\\

\begin{tabular}{|l|p{15cm}|}
\hline
\hline
Class name & Bool_wildcard extends Expression \\
\hline
Parsing & 
\begin{grammar}
<bool_wildcard> ::=  `_Bool'
\end{grammar}  \\
\hline
Typing &  Bool
\\
\hline
Evaluation &  has no evaluation rules \\
\hline
Pretty printing &  $\_Bool$  has string  ``it: Bool it = _Char" \\
\hline
\end{tabular} 
\\
\begin{tabular}{|l|p{15cm}|}
\hline
\hline
Class name & If_Statement extends Expression \\
\hline
Parsing & 
\begin{grammar}
<if_statement> ::= `if' <exp> `then' <exp> pTerm `else' <exp>
\end{grammar}
 \\
\hline
Typing &  \infer[]
       {\Gamma \; \vdash \text{if} \; t_1 \text{ then } t_2 \text{ else } t_3 : T } 
       {\Gamma \; \vdash t_1:Bool \;\; \Gamma \; \vdash t_2:T \;\; \Gamma \; \vdash t_3:T}
\\
\hline
Evaluation & $if\_statment$ reduces to $exp$ \\
\hline
Pretty printing &  Once evaluated  has string  ``it: T it = value_exp" \\
\hline
\end{tabular} 

\end{tabular}

\subsection{let-terms (with their status)}


\begin{tabular}{|l|p{15cm}|}
\hline
\hline
Class name & Let extends Expression \\
\hline
Parsing & 
\begin{grammar}
<let_exp> ::= `let' ((<let_status>  <binding>) | <binding_list>) `in' <exp>

<let_status> ::= `rec' | `ext' | `method' | `discontinuous' | `linear'

<binding> ::= <parameter> <simple_exp>* `=' <exp>

<parameter> ::= <term_id> | <parameter> `:' <type> | `(' <parameter> `)' | `(' <infix_op> `)' 
\end{grammar}
 \\
\hline
Typing & \infer[]
       {\Gamma \; \vdash \text{let} x=t_1 \text{ in } t_2:T_2} 
       {\Gamma \; \vdash t_1:T_1  \Gamma,x:T_1 \; \vdash t_2:T_2} \\
\hline
Evaluation & $let\_exp$ reduces to $exp$ if it has all its arguments, remains unevaluated otherwise \\
\hline
Pretty printing &  Once evaluated  has string  ``it: T it = value_exp" \\
\hline
\end{tabular} 


\subsection{Dynamic patterns }

\begin{tabular}{|l|p{15cm}|}
\hline
\hline
Class name & Dynamic_case extends Case \\
\hline
Parsing & 
\begin{grammar}
<dynamiccase> :: =  `|' `{' <parameterCommaList> `}' <exp>  `->'  <exp> 

<parameterCommaList> ::= <lident> | <lident> `,' <parameterCommaList> 
\end{grammar}
 \\
\hline
Typing & $p \rightarrow s$\\
\hline
Evaluation &  \\
\hline
Pretty printing &   ``it = $T_1 \rightarrow T_2$" \\
\hline
\end{tabular} 



\subsection{References }
\bondi\ references have values that can be read and written. References to references
can be used as pointers.

\begin{verbatim}
~~ let x = Ref 3;;
x: ref Int
~~ !x;;
it: Int
it = 3
~~ x = !x + 1;;
it: Unit
~~ !x;;
it: Int
it = 4
\end{verbatim}

The example above, creates a reference in the store, where a value $3$ is placed under the entry $x$.
The value of $x$ can be read by using the $!$ operator as in $!x$, in the example
$!x$ returns the value $it = 3$.
A reference can be written by using the $=$ operator as for example
the code $x = !x + 1;;$  writes the value of the expression $!x + 1$ into
the value pointed by $x$.


It is also possible to pattern match over references as they are constructors; for example:

\begin{verbatim}
 ~~ select (is _Int) (Ref 1,4,5)
;;
it: List Int
it = [1,4,5]
\end{verbatim}


Creating,reading and writing references can be split into three classes for three operators:   

\subsubsection{Ref constructor}
Class: Ref extends Expression

\begin{tabular}{|l|p{15cm}|}
\hline
Parsing & \begin{grammar}
<reference> ::= `Ref' <Expression>   
\end{grammar}  \\
\hline
Typing & $Ref exp\; \text{has type} \; \forall X.X \rightarrow ref \;X$
\\
\hline
Evaluation &  Ref is a constructor \\
\hline
Pretty printing & $Ref \; exp\; \text{has string} \;``ref  (toString \; T)$ \\
\hline
\end{tabular} 


\subsubsection{Assignment}

Class: Assigment extends Action

\begin{tabular}{|l|p{15cm}|}
\hline
Parsing & \begin{grammar}
<assignment> ::= <expression> `=' <expression> 
\end{grammar}  \\
\hline
Typing & \infer[^{if \;T_2 < T_1}]
       {\Gamma \; \vdash t_1=t_2:Unit} 
       {\Gamma \; \vdash t_1:ref\;T_1 \;\; \Gamma \; \vdash t_2:T_2}
\\
\hline
Evaluation & $t  \rightharpoonup Unit$ \\
\hline
Pretty printing &  \\
\hline
\end{tabular} 

\subsubsection{Dereference}

Class: Dereference extends Operator

\begin{tabular}{|l|p{15cm}|}
\hline
Parsing & \begin{grammar}
<!> ::= `!' <expression> 
\end{grammar}  \\
\hline
Typing & \infer[^{}]
       {\Gamma \; \vdash \; !t_1:T_1}
       {\Gamma \; \vdash \; t_1:ref \; T_1} 
\\
\hline
Evaluation & $!e  \rightharpoonup  deref\;\; e$ \\
\hline
Pretty printing &  !($x:ref\;T$) has string  $toString \; T$\\
\hline
\end{tabular} 
\subsection{Arrays (including while- and for-loops) }

In computer science, an array data structure or simply array is a data structure consisting of a collection of elements (values or variables), 
each identified by at least one index. An array is stored so that the position of each element can be computed from its index.

In \bondi\ arrays can be used as follows:

\begin{verbatim}
~~ let array1 = newarray 0 5;;
array1: array Int
array1 = ...
~~ lengthv array1;;
it: Int
it = 5
~~ entry(array1,0);;
it: ref Int
!(entry(array1,0));;
it: Int
it = 0
~~ (entry(array1,0)) = 7;;
it: Unit
~~ !(entry(array1,0));;
it: Int
it = 7
\end{verbatim}

An array is created using the \verb@newarray@ operator that receives two parameter, the first parameter is the default value for initialization and 
the second parameter is the array length.
The \verb@lengthv@ operator returns an Int value, that represents the array length.
The elements of the array can be accessed by using the \verb@entry@ operator; the \verb@entry@ operator has two parameters, the first parameter 
is the array whose values we want to access, the second parameter is the index of an array location.
The \verb@entry@ operator can be used with the operator for assigment \verb@=@ to change the value of an array location, and also the same \verb@entry@ operator
can be used with the dereference operator \verb@!@ to read the value of an array location.

\subsubsection{newarray}

Class: NewArray extends operator

\begin{tabular}{|l|p{15cm}|}
\hline
Parsing & \begin{grammar}
<newarray> ::= `newarray' <type> <expression>   
\end{grammar}  \\
\hline
Typing & newarray ty int \; \text{has type} \; $\forall X.X \rightarrow Int \rightarrow  array \;X$
\\
\hline
Evaluation &   \\
\hline
Pretty printing & newarray ty int \text{has string} $\;``array  (toString \; T)$ \\
\hline
\end{tabular} 


\subsubsection{lengthv}

Class: NewArray extends Operator

\begin{tabular}{|l|p{15cm}|}
\hline
Parsing & \begin{grammar}
<lengthv> ::= `lengthv' <expression>   
\end{grammar}  \\
\hline
Typing & $lengthv$ \text{ has type} \; $array \; a \rightarrow Int$
\\
\hline
Evaluation &  $lengthv \; e \; \rightharpoonup e$\\
\hline
Pretty printing & $newarray \;ty \; int$ \text{has string} $\;``array  (toString \; T)$ \\
\hline
\end{tabular} 

\subsubsection{entry}
Class: Entry extends Operator

\begin{tabular}{|l|p{15cm}|}
\hline
Parsing & \begin{grammar}
<entry> ::= `entry' <expression> <expression>   
\end{grammar}  \\
\hline
Typing & $entry$ \; \text{has type} \; $array \;a \rightarrow Int \rightarrow Ref\; a$
\\
\hline
Evaluation &  $entry \;array\; int \; \rightharpoonup Ref\; e$\\
\hline
Pretty printing & $newarray \;ty \; int$ \text{has string} $\;``array" (toString \; T)$ \\
\hline
\end{tabular} 


\subsubsection{while-loop}


\subsubsection{for-loop}



\subsection{Sockets (including file IO)}

Sockets in bondi provide network conectivity and file I/O.
Socket and Host are two constant types, like Int, Float, Char, etc...
Values of type socket and Host are created by Operators.


\begin{itemize}
 \item  Execute the given command, wait until it terminates, and return its termination status \textbf{sysexec}
 \item  Returns a term of type host \textbf{gethost}
 \item  Open the named file on a socket, returns the socket   \textbf{openfile}
 \item  Opens a connection to a server on host h and port p, and returns a socket  \textbf{openserver}
 \item  Opens a tcp connection to host h and port p and returns a socket \textbf{opentcp}
 \item  Reads a line of text from a socket \textbf{readline}
 \item  Reads all the content of a socket buffer as text \textbf{readall}
 \item  Accepts a connection to an open socket \textbf{acceptclient}
 \item  Close a socket \textbf{close}
 \item  Writes to a socket {write}
 \item  Writes a line of text to a socket {writeline}
\end{itemize}                                                                                  


\subsection{Concurrent processes (cpc) (this is hard, and requires sockets)  }

In computer science, the process calculi (or process algebras) are a diverse family of related approaches to formally modeling concurrent systems. 
Process calculi provide a tool for the high-level description of interactions, communications, and synchronizations between a collection of independent agents or processes.

The concurrent pattern calculus generalizes the $\Pi$-calculus by replacing the input and output prefixes
by patterns and with interaction driven by pattern unification. The resulting calculus supports exchange
while supporting encodings of many popular process calculi such as $Pi$-calculus, spi calculus and
Linda.

More information on CPC can be found in \cite{}


\section{Shell Actions}

\subsection{Let-declarations (with let-status) }
\begin{tabular}{|l|p{15cm}|}
\hline
\hline
Class name & Let extends Action \\
\hline
Parsing & 
\begin{grammar}
<let_exp> ::= `let' ((<let_status>  <binding>) | <binding_list>) `in' <exp>

<let_status> ::= `rec' | `ext' | `method' | `discontinuous' | 'linear'

<binding> ::= <parameter> <simple_exp>* `=' <exp>

<parameter> ::= <term_id> | <parameter> `:' <type> | `(' <parameter> `)' | `(' <infix_op> `)' 
\end{grammar}
 \\
\hline
Typing & \\
\hline
Evaluation &  \\
\hline
Pretty printing &  Once evaluated  has string  ``it: T it = $T_1 -> T_2$" \\
\hline
\end{tabular} 
\subsection{Adding a case}
Class: Add_case extends Action

\begin{tabular}{|l|p{15cm}|}
\hline
Parsing & \begin{grammar}
<addCase> ::=  (<Lident> | <Uident> `.' <Lident>) `+=' <Case>   
\end{grammar}  \\
\hline
Typing & 
\\
\hline
Evaluation &  \\
\hline
Pretty printing &  \\
\hline
\end{tabular} 
\subsection{Type synonyms}
\section{Datatype declarations}
Class: Let_type

\begin{tabular}{|l|p{15cm}|}
\hline
Parsing & \begin{grammar}
<Let_type> ::= `datatype' ( <Uident> | <infixOp> ) `=' <constructor_info> <trailer>? 

<constructor_info> ::= <constructor> (`|' <constructor>) *

<constructor> ::= <Uident> | <Uident> `of' <constructor_type>

<constructor_type> ::= <type> | <constructor_type> `and' <constructor_type>
\end{grammar}  
   \\
\hline
Typing & 
\\
\hline
Evaluation  \\
\hline
Pretty printing &  \\
\hline
\end{tabular} 
\subsection{Class declarations}
\subsubsection{Method}
Class: Method extends Expression

\begin{tabular}{|l|p{15cm}|}
\hline
Parsing & \begin{grammar}
<method> ::= <Lident> `=' `{' <expression> `}'
\alt `static' (`rec' | `ext' | `discontinous')* `=' `{' <expression>  <trailer>? `}'

\end{grammar}  
   \\
\hline
Typing & 
\\
\hline
Evaluation &  \\
\hline
Pretty printing &  \\
\hline
\end{tabular} 

Class: Class

\begin{tabular}{|l|p{15cm}|}
\hline
Parsing & \begin{grammar}
<Class> ::= `Class' <Uident> (`<' <parameterCommaList> `>' )?  `{' <attributeList> `}'  

<attributeList> ::= (<field>|<method>)*

<field> ::= (<term_id> | <parameter>) `;'
\end{grammar}  
   \\
\hline
Typing & 
\\
\hline
Evaluation &  \\
\hline
Pretty printing &  \\
\hline
\end{tabular}

\subsection{Class declarations  (including static methods and generic types)}
\subsection{Augmenting declarations with trailing actions (with ... and ...)}
Class: Trailer

\begin{tabular}{|l|p{15cm}|}
\hline
Parsing & \begin{grammar}
<trailer> ::= `with' <action_list>
<action_list> ::= <action> | <action> `and' <action_list>
\end{grammar}  
   \\
\hline
Typing & 
\\
\hline
Evaluation  \\
\hline
Pretty printing &  \\
\hline
\end{tabular} 

\section{The Context}

Context is a class of the combinatory language that is used in type
inference.  We plan to extend the context class with the various
global environments.  There is also a class of Actions which will be
extended, too.

\begin{program}
@class Context {@ ... \\ 
@act = { fun action -> action.do(this) }@  \\
... \\
@}@ 
\end{program}
allows for specialisation of both classes. 

The existing context must be
extended to support the following fields

\subsection{globalTyEnv: ref (type_env)}
\subsection{globalVEnv: ref (global_value_env) }
\subsection{globalCEnv: ref (constructor_env) }
\subsection{globalRefVarSub: ref sub where sub is a TyMap that produces types}


\section{Data Sructures and Algorithms }
\subsection{hash tables}
\subsection{balanced search trees}
\subsection{garbage collection (hard)} 

\section{Auxiliary Programs}
Here are features that are {\em not} within the natural class structure:

\subsection{parsing of comments, infix operators, tuples and lists, etc}
\subsection{ the unit type, exceptions, etc }



\section{Integration and Testing}
Integration and testing must work with the whole project team.  Marks
will be the average of the class as a whole, adjusted according to
feedback, or by negotiation. 

\subsection{integration}
\subsection{testing}


\bibliographystyle{plain}
\bibliography{dsl}

\end{document}
